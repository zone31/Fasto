\documentclass[11pt]{article}
\usepackage[a4paper, hmargin={2.8cm, 2.8cm}, vmargin={2.5cm, 2.5cm}]{geometry}
\usepackage{eso-pic} % \AddToShipoutPicture
\usepackage{graphicx} % \includegraphics
\usepackage{fancyhdr, amsmath, amssymb, comment, caption, placeins, subfigure,
    fixltx2e, changepage, listings, courier, soul, hyperref, geometry,
    enumerate, dsfont, listings, enumitem}
    \usepackage[T1]{fontenc}

\author{\Large{Magnus N\o rskov Stavngaard} \\
		\texttt{magnus@stavngaard.dk}
		\\\\
		\Large{Mark Jan Jacobi} \\
        \texttt{mark@jacobi.pm}
		 \\\\
		\Large{Christian Salb\ae k} \\
		\texttt{chr.salbaek@gmail.com}
}

\title{
    \vspace{3cm}
    \Huge{Compiler Design} \\
    \Large{Compiler for the Fasto Programming Language}
}

\pagestyle{fancy}
\lhead{\small{Magnus N. S. Mark J. J. Christian S.}}
\chead{\date{\today}}
\rhead{University of Copenhagen}
% \lfoot{}
% \cfoot{}
% \rfoot{}

% Change indent length of paragraph not after a header.
\setlength{\parindent}{0cm}

% Remove page numbering in the beginning
\pagenumbering{gobble}

\begin{document}

    %% Change `ku-farve` to `nat-farve` to use SCIENCE's old colors or
    %% `natbio-farve` to use SCIENCE's new colors and logo.
    \AddToShipoutPicture*{\put(0,0){\includegraphics*[viewport=0 0 700 600]
        {include/ku-farve}}}
    \AddToShipoutPicture*{\put(0,602){\includegraphics*[viewport=0 600 700 1600]
        {include/ku-farve}}}

    %% Change `ku-en` to `nat-en` to use the `Faculty of Science` header
    \AddToShipoutPicture*{\put(0,0){\includegraphics*{include/ku-en}}}

    \clearpage
    \maketitle
    \thispagestyle{empty}

    \newpage

    \tableofcontents

    \newpage

    \pagenumbering{arabic} % Arabic page numbers (and reset to 1)

    \section{Task 1 - Warmup}
    In task 1 we were asked to implement the boolean operators \texttt{\&\&},
    \texttt{||} and \texttt{not}, the boolean constants true and false, integer
    multiplication, integer division and integer negation. \\

    We will go through in detail the implementation of integer division and
    multiplication and then skip rather quickly over the implementation of the
    rest of the operators only describing what is different from multiplication
    and division as the operations is implemented very similar.

    \subsection{Integer Multiplication and Division}
    We started by implementing integer multiplication and division in the Lexer.
    We created a new rule for the star and division operator that created tokens
    and passed the tokens to the parser.

    \begin{lstlisting}[basicstyle=\small]
| `*`   { Parser.TIMES    (getPos lexbuf) }
| `/`   { Parser.DIVIDE   (getPos lexbuf) }
    \end{lstlisting}

    In the parser we added the tokens where addition and subtraction was already
    defined, as integer multiplication and division has allot in common with
    addition and subtraction.  Integer multiplication and division carries two
    integers corresponding to a position in the code.

    \begin{lstlisting}[basicstyle=\small]
%token <(int*int)> PLUS MINUS TIMES DIVIDE DEQ EQ LTH BOOLAND BOOLOR NOT NEG
    \end{lstlisting}

    We also declare both times and divide as left associative operators with
    greater precedence than addition and subtraction.

    \begin{lstlisting}[basicstyle=\small]
%left BOOLOR
%left BOOLAND
%left NOT
%left DEQ LTH
%left PLUS MINUS
%left TIMES DIVIDE
%left NEG
    \end{lstlisting}

    We then defined that an expression could consist of an expression followed
    by a multiplication or division followed by an expression.  And that this
    correspond to \texttt{Times} and \texttt{Divide} in the Fasto language
    definition.

    \begin{lstlisting}[basicstyle=\small]
Exp :     NUM                 { Constant (IntVal (#1 $1), #2 $1) }
        | CHARLIT             { Constant (CharVal (#1 $1), #2 $1) }

        (...)

        | Exp TIMES Exp       { Times($1, $3, $2) }
        | Exp DIVIDE Exp      { Divide($1, $3, $2) }

        (...)
    \end{lstlisting}

    In the interpreter we implemented cases for \texttt{Times} and
    \texttt{Divide} in the \texttt{evalExpr} function.

    \begin{lstlisting}[basicstyle=\small]
| evalExp ( Times(e1, e2, pos), vtab, ftab ) =
      let val res1 = evalExp(e1, vtab, ftab)
          val res2 = evalExp(e2, vtab, ftab)
      in  evalBinopNum(op *, res1, res2, pos)
      end

| evalExp ( Divide(e1, e2, pos), vtab, ftab ) =
      let val res1 = evalExp(e1, vtab, ftab)
          val res2 = evalExp(e2, vtab, ftab)
      in  evalBinopNum(op Int.quot, res1, res2, pos)
      end
    \end{lstlisting}

    Our cases evaluate recursively the expressions to the left and right of the
    operator and then calls \texttt{evalBinopNum} with the appropriate operator
    and the results from evaluating the lefthand and the righthand side of the
    expression. \\

    We then implemented the operators in the typechecker.  Our cases call a
    helper function \texttt{checkBinOp} that takes a position, an expected type,
    and two expressions and check that the two expressions have the type of the
    expected type.  If the types match the types is returned with
    \textit{typedecorated} versions of the expressions, if the types doesn't
    match an error is raised. \\
    We then simply return the same operation, now with a return type.

    \begin{lstlisting}[basicstyle=\small]
| In.Times (e1, e2, pos)
  => let val (_, e1_dec, e2_dec) = checkBinOp ftab vtab (pos, Int, e1, e2)
     in (Int, Out.Times (e1_dec, e2_dec, pos))
     end

| In.Divide (e1, e2, pos)
  => let val (_, e1_dec, e2_dec) = checkBinOp ftab vtab (pos, Int, e1, e2)
     in (Int, Out.Divide (e1_dec, e2_dec, pos))
     end
    \end{lstlisting}

    We can now finally implement the operators in the code generator.  Here we
    create two temporary variables \texttt{t1} and \texttt{t2} to hold the
    values of the expression on either side of the operator.  We then call the
    function \texttt{compileExp} recursively with these names to get the machine
    code for the expression on either side of the operator.  Then we just simply
    return a list of first the code to compute the left hand side of the
    operator, then the right hand side, and then we apply the MIPS commands
    \texttt{MUL} and \texttt{DIV} to the two \textit{subresults} and save the
    result in \texttt{place}.

    \begin{lstlisting}[basicstyle=\small]
| Times (e1, e2, pos) =>
    let val t1 = newName "times_L"
        val t2 = newName "times_R"
        val code1 = compileExp e1 vtable t1
        val code2 = compileExp e2 vtable t2
    in  code1 @ code2 @ [Mips.MUL (place, t1, t2)]
    end
| Divide (e1, e2, pos) =>
    let val t1 = newName "divide_L"
        val t2 = newName "divide_R"
        val code1 = compileExp e1 vtable t1
        val code2 = compileExp e2 vtable t2
    in  code1 @ code2 @ [Mips.DIV (place, t1, t2)]
    end
    \end{lstlisting}

    \subsection{Boolean Operators}
    We started by implementing the boolean operators in the lexer in a very
    similar way that we implemented multiplication and division.  In the parser,
    we made sure that the boolean operators were defined as having a lower
    precedence than the arithmetic operators, so that an expression like,

    \begin{lstlisting}[basicstyle=\small]
        2 + 4 == 6 || 5 + 8 == 10 && 8 < 10
    \end{lstlisting}

    is evaluated like,

    \begin{lstlisting}[basicstyle=\small]
        ((2 + 4) == 6) || (((5 + 8) == 10) && (8 < 10)).
    \end{lstlisting}

    Notice that the \texttt{\&\&} operator is also evaluated before the
    \texttt{||} operator. \\

    After this we implemented the operators in the interpreter.  Here we created
    a case for \texttt{and} and a case for \texttt{or}.  We had to implement
    them as short circuited which means that the right hand side of an
    \texttt{and} should only be evaluated if the left hand side is true.
    Similarly for \texttt{or} the right hand side should only be evaluated if
    the left hand side is false (The or implementation can be seen in appendix
    \ref{interpreter_and_and_or}.

    \begin{lstlisting}[basicstyle=\small]
| evalExp ( And(e1, e2, pos), vtab, ftab ) =
      let val r1 = evalExp(e1, vtab, ftab)
      in case r1 of
         BoolVal b1 => if b1 then
                        let val r2 = evalExp(e2, vtab, ftab)
                        in case r2 of
                           BoolVal b2 => BoolVal b2
                         | otherwise  => raise Error ("And expect boolval", pos)
                        end
                       else BoolVal b1
       | otherwise  => raise Error ("And expect boolval", pos)
      end

    \end{lstlisting}

    We first evaluate the lefthand expression, if that is true we evaluate the
    right hand expression, if that is also true we return True otherwise we
    return false.  If either of the expressions isn't a BoolVal, we report an
    error. \\

    In the code generator it was also required that we implemented the boolean
    operators to be short-circuiting so that the right hand side of an
    \texttt{and} only evaluates if the left hand side is true.  Similarly the
    right hand side of an \texttt{or} evaluates only if the left hand side is
    false.  We did this with the MIPS assembly code,

    \begin{lstlisting}[basicstyle=\small]
$t1 = compile e1

beq $t1, $zero, False

$t2 = compile e2

beq $t2, $zero, False
li  $s1, 1 ;; Assuming the result should be saved to $s1
j   End

False:
    li  $s1, 0 ;; Assuming the result should be saved to $s2

End:
    \end{lstlisting}

    It can be seen that if the first part of an \texttt{and} return false i.e.
    the result register contains 0, we simply skip over the execution of the
    right hand side and jump strait to False.  We did something similar for
    \texttt{or}'s.  The Standard ML code generating the MIPS assembly for both
    \texttt{and} and \texttt{or} can be seen in appendix
    \ref{code_gen_and_and_or}.

    \subsection{Boolean Negation}
    In the lexer we implemented the operator for boolean negation \texttt{not}
    as a keyword.  We did this because \texttt{not} is a valid variable name and
    would pass the rule,

    \begin{lstlisting}[basicstyle=\small]
| [`a`-`z` `A`-`Z`] [`a`-`z` `A`-`Z` `0`-`9` `_`]*
                      { keyword (getLexeme lexbuf,getPos lexbuf) }.
    \end{lstlisting}

    If we didn't implement a keyword, \texttt{not} would simply fall through and
    go in the case,

    \begin{lstlisting}
| _              => Parser.ID (s, pos)
    \end{lstlisting}

    in the \texttt{keyword} function.  The implemented keyword goes to a
    \texttt{Parser.NOT} in the parser and carries only the position.

    \begin{lstlisting}[basicstyle=\small]
fun keyword (s, pos) =
    case s of
         "if"           => Parser.IF pos
       | "then"         => Parser.THEN pos

       (...)

       | "not"          => Parser.NOT pos
       | "fn"           => Parser.LAMBDA pos
       | _              => Parser.ID (s, pos)
}
    \end{lstlisting}

    In the interpreter \texttt{not} is implemented simply by evaluating the
    expression after the \texttt{not}.  If that expression results in a true,
    false is returned, if it results in a false, true is returned and otherwise
    an error is reported.  The code can be seen in appendix
    \ref{interpreter_not}. \\

    \texttt{not} is implemented in the code generator as a branch operation.  If
    \texttt{not} is applied to true 3 operations are performed, if it is applied
    to false only 2 operations are performed.  The code generating MIPS assembly
    can be found in appendix \ref{code_gen_not}.

    \subsection{Integer Negation}
    Integer negation has been implemented two places in the lexer.  For integer
    constants which is negated the negated value is simply created on
    compiletime.  This is done by the rule,

    \begin{lstlisting}[basicstyle=\small]
| [`0`-`9`]+ | "~" [`0`-`9`]+ { case Int.fromString (getLexeme lexbuf) of
                             NONE   => lexerError lexbuf "Bad integer"
                           | SOME i => Parser.NUM (i, getPos lexbuf) }
    \end{lstlisting}

    which says that a number or a tilde followed by a one or more numbers is a
    Parser.NUM in the parser and the integer value is carried with it. \\

    If a tilde is not followed by a number for example in,

    \begin{lstlisting}
~(2 + 4) - ~(2 - ~f(3))
    \end{lstlisting}

    the negation is caught by the rule,

    \begin{lstlisting}
| "~"                 { Parser.NEG      (getPos lexbuf) }
    \end{lstlisting}

    In the code generator the negations is performed as a exclusive or and an
    addition.

    \begin{lstlisting}[basicstyle=\small]
| Negate (e, pos) =>
    let val negThis  = newName "negThis"
        val code     = compileExp e vtable negThis
        val negation =
            [Mips.XORI(negThis,negThis,"-1")] @ [Mips.ADDI (place,negThis,"1")]
    in code @ negation
    end
    \end{lstlisting}

    This works because computers use two's complement to express numbers.  In
    two's complement the negation of a number is computed by flipping all bits
    in the number and adding 1.  We flip all bits by exclusive or'ring with -1,
    the binary value of -1 is 1111 1111.  Therefore all the places where there
    was 0 in the original number there will now be 1 and where there were 1
    there will now be 0.  After that we simply add 1 with \texttt{addi} and save
    the result to place.

    \subsection{Boolean Literals}
    The boolean literals is implemented in the lexer as a keyword that carries a
    boolean value to the parser.  The implementation of boolean literals in all
    compiler phases can be found in appendix \ref{boolean_implementation}.

    \subsection{Test}
    \begin{tabular}{|l|l|}
        \hline
        \textbf{Test name} & \textbf{Test description}                        \\
        \hline
        andOr.fo           & Tests boolean operators and their precedence.    \\
        \hline
        and\_sc.fo         & Test if and is properly short circuited.         \\
        \hline
        boolCompare.fo     & Test boolean literals.                           \\
        \hline
        boolLit.fo         & Another test testing if boolean literals.        \\
        \hline
        intNegate.fo       & Test negation of integer values, only test the   \\
                           & simple case ~number not ~(number).  Operator     \\
                           & precedence is also tested.                       \\
        \hline
        muldivide.fo       & Tests the multiplication and division operators. \\
        \hline
        negate2.fo         & Again test simple integer negation.              \\
        \hline
        negate.fo          & Test the case where integer negation is followed \\
                           & by something other than a number.                \\
        \hline
        or\_sc.fo          & Test if or is short circuited.                   \\
        \hline
        tobeornottobe.fo   & Test precedence of boolean operators.            \\
        \hline
        and\_err.fo        & Test a typeerror for and.                        \\
        \hline
        div\_err.fo        & Test a typeerror for div.                        \\
        \hline
        mul\_err.fo        & Test a typeerror for mul.                        \\
        \hline
        neg\_err           & Test a typeerror for negation.                   \\
        \hline
        not\_err.fo        & Test a typeerror for not.                        \\
        \hline
        or\_err.fo         & Test a typeerror for or.                         \\
        \hline
    \end{tabular}

    All the test files are included in appendix \ref{task_1_tests}.

    \section{Task 2 - Implement \texttt{filter} and \texttt{scan}}
    \subsection{Lexer and Parser}
    \texttt{Filter} and \texttt{scan} is implemented in the lexer simply as
    keywords.\\

    In the parser \texttt{scan} and \texttt{filter} is implemented as,

    \begin{lstlisting}[basicstyle=\small]
| FILTER LPAR FunArg COMMA Exp RPAR { Filter ($3, $5, (), $1) }
| SCAN LPAR FunArg COMMA Exp COMMA Exp RPAR { Scan ($3, $5, $7, (), $1) }
    \end{lstlisting}

    Where a \texttt{FUNARG} is,

    \begin{lstlisting}[basicstyle=\small]
FunArg : ID { FunName (#1 $1) }
     |   LAMBDA Type LPAR Params RPAR LAMBDAEQ Exp { Lambda ($2, $4, $7, $1) }
     |   LAMBDA Type LPAR RPAR LAMBDAEQ Exp { Lambda ($2, [], $6, $1) }
    \end{lstlisting}

    This means that \texttt{filter} consist of a keyword filter followed by a
    left parenthesis, followed by a function argument (either a function name or
    a lambda), followed by another comma and at last a right parenthesis.  The
    expression, the function argument and the position is given to
    \texttt{filter} in the fast language.

    \subsection{Typechecker}
    The typerules for \texttt{filter} and \texttt{scan} have been based on the
    already existing typerules for \texttt{map} and \texttt{reduce}. They are as
    follows: \\

    \texttt{filter}: ($\alpha \rightarrow$ bool) $\ast$ $[\alpha]$ $\rightarrow$
    $[\alpha]$, typerule for \texttt{filter(f, x)}:
    \vspace{-2.5mm}
    \begin{itemize}[noitemsep]
        \item compute t, the type of x and check that $\text{t} =
            [\text{t}_\text{e}]$ for some type $\text{t}_\text{e}$
        \item get f's signature from \texttt{ftable}. \texttt{IF} f does not
            receive exactly one argument \texttt{THEN} return \texttt{error()}
            \texttt{ELSE} $\text{f: t}_{\text{in}} \rightarrow
            \text{t}_{\text{out}}$ for some types $\text{t}_{\text{in}}$ and
            $\text{t}_{\text{out}}$.
        \item \texttt{IF} $\text{t}_{\text{in}} = \text{t}_\text{e}$
            \texttt{AND} $\text{t}_{\text{out}} = \texttt{bool}$ \texttt{THEN
            filter(f, x) ELSE error()}.
    \end{itemize}

    \texttt{scan}: ($\alpha \ast \alpha \rightarrow \alpha$) $\ast$ $\alpha$
    $\ast$ [$\alpha$] $\rightarrow$ [$\alpha$], typerule for
    \texttt{scan(f, e, x)}:
    \vspace{-2.5mm}
    \begin{itemize}[noitemsep]
        \item Compute t, the type of e and $\text{t}_\text{x}$, the type of x
            and check that:
        	\begin{enumerate}[noitemsep]
        	   \item f: (t $\ast$ t) $\rightarrow$ t
        	   \item $\text{t}_\text{x} = [\text{t}]$
    		\end{enumerate}
        \item If so then \texttt{scan(f, e, x)}
    \end{itemize}

    We will here show the implementation of the type rule of filter, the
    typerule of scan implementation can be seen in appendix
    \ref{interpreter_scan}.

    \begin{lstlisting}[basicstyle=\small]
| In.Filter (f, arr_exp, _, pos)
  => let val (arr_type, arr_exp_dec) = checkExp ftab vtab arr_exp
         val elem_type =
           case arr_type of
               Array t => t
             | other   => raise Error ("Filter: Argument not an array", pos)
         val (f', f_res_type, f_arg_type) =
           case checkFunArg (f, vtab, ftab, pos) of
               (f', res, [a1]) => (f', res, a1)
             | (_,  res, args) =>
               raise Error ("Filter: incompatible function type of "
                            ^ In.ppFunArg 0 f ^ ":" ^ showFunType (args, res), pos)
     in if elem_type = f_arg_type andalso f_res_type = Bool
        then (arr_type,
              Out.Filter (f', arr_exp_dec, elem_type, pos))
        else raise Error ("Filter: array element types does not match."
                          ^ ppType elem_type ^ " instead of "
                          ^ ppType f_arg_type , pos)
     end
    \end{lstlisting}

    It can be seen that the typechecker raises an error if the second argument
    of filter is not an array or if the function doesn't have the correct type.

    \subsection{Code Generation}
    In the code generator \texttt{filter} and \texttt{scan} is implemented in
    two different ways depending on what types is in the array.  A
    \textit{Fasto} array changes the size of the types depending on if it
    consists of booleans, chars or integers.  An integer is represented with 4
    bytes in \textit{Fasto} while a boolean and a char is represented with only
    one byte. \\

    We will here in detail go through the implementation of \texttt{scan}, the
    complete Standard ML code generating MIPS assembly for both \texttt{scan}
    and \texttt{filter} can be found in appendix \ref{code_gen_scan_filter}.
    The code for scan is put together by,

    \begin{lstlisting}[basicstyle=\small]
| Scan (farg, acc_exp, arr_exp, tp, pos) =>
  let val arr_reg   = newName "arr_reg"   (* Address of array given. *)

      (...)

  in arr_code  @ (* Compute input array expr. *)
     get_size  @ (* Save size of input array to size_reg. *)
     comp_size @ (* Calculate size of new arr. *)
     neu_code  @ (* Compute the neutral element. *)
     dynalloc (new_size, place, tp) @ (* Allocates space for res. *)
     set_r_w   @ (* Set read and write registers. *)
     whileinit @ (* While initialization. *)
     whilecond @ (* Test the while condition. *)
     whileloop @ (* Code of while. *)
     while_end   (* While end label. *)
  end.
    \end{lstlisting}

    We will go through each part of the resulting list, explaining what each
    list item consist of.

    \begin{description}
        \item [arr\_code] Here we simply call the function \texttt{compileExpr}
            recursively computing the array to run \texttt{scan} over and saving
            that array to a name, \texttt{arr\_reg}.

        \item [get\_size] Here we save the size of the start array to a name
            \texttt{size\_reg} by loading the first word from the input array.

        \item [comp\_size] We compute the length of the resulting array by
            adding one to the size of the old array (space for neutral element).
            This new size we save to a name \texttt{new\_size}.

        \item [neu\_code] We compute the value of the neutral element by calling
            the function \texttt{compileExpr} recursively on \texttt{acc\_exp}.

        \item [dynalloc] We call the function \texttt{dynalloc} to allocate
            space for the resulting array.  A pointer to the allocated space is
            saved to place, which is where we want to save our result.

        \item [set\_r\_w] Here we initialize the pointers to where to read data
            from the old array and write data to the resulting array.

            \begin{lstlisting}[basicstyle=\small]
val set_r_w   = case getElemSize tp of
                  One  => [ Mips.ADDI (elreadreg, arr_reg, "4")
                          , Mips.ADDI (elwritreg, place, "5") ]
                | Four => [ Mips.ADDI (elreadreg, arr_reg, "4")
                          , Mips.ADDI (elwritreg, place, "8") ]
            \end{lstlisting}

            The read register is always the address of the start array plus 4
            bytes as the first word in an array is the length of the array. \\

            The place to write changes with the type of the resulting array, if
            the size of the type is one byte the first element should be written
            5 bytes after the start of the array.  The byte at place 4 is the
            neutral element.  If the size of the type is 4 bytes the first write
            should be 8 bytes from the start of the array to make space for
            length and the neutral element.

        \item [whileinit] Here i is set to 0 and the neutral element is placed
            in the array, either with \texttt{Mips.SB} or \texttt{Mips.SW}
            depending on the array type.

        \item [whilecond] The while condition is checked.  If i is equal to the
            size of the start array we jump to \texttt{while\_end}, otherwise
            we continue computing.

        \item [whileloop] This is the main part of the while loop, here we
            present only the code for integers, chars and booleans are very
            similar.

            \begin{lstlisting}[basicstyle=\small]
val whileloop = case getElemSize tp of
                | Four => [ Mips.LW (res_reg, elreadreg, "0")
                          , Mips.LW (tmp_reg, elwritreg, "-4") ]
                          @ applyFunArg(farg, [tmp_reg, res_reg],
                                        vtable, res_reg, pos)
                          @ [ Mips.SW (res_reg, elwritreg, "0")
                          , Mips.ADDI (elreadreg, elreadreg, "4")
                          , Mips.ADDI (elwritreg, elwritreg, "4")
                          , Mips.ADDI (i_reg, i_reg, "1")
                          , Mips.J startlabl ]
            \end{lstlisting}

            First we load an integer from the initial array, then we load the
            previously computed integer by loading a word from the result array
            with the ofset -4.  Then we apply the function argument to the two
            loaded values and save the value to \texttt{res\_reg}.  Then we save
            that value to the resulting array and increment all pointers and
            counter variables.  Lastly we jump to the while condition again.

        \item [while\_end] While end is simply a label to jump out of the while
            loop.
    \end{description}

    \subsection{Test}
    \begin{tabular}{|l|l|}
        \hline
        \textbf{Test name} & \textbf{Test description}                        \\
        \hline

    \end{tabular}


    \section{Task 3 - $\lambda$-expressions in SOAC's}
    \subsection{Lexer}
    We implemented lambda functions in the lexer by creating a keyword
    \texttt{fn} corresponding to a \texttt{LAMBDA} in the parser.  We also
    created a token for the special equals symbols used in lambda expressions
    (=$>$).  We called this token \texttt{LAMBDAEQ}.
    \subsection{Parser}
    We then implemented lambdas
    in the parser.  We did this by observing that the \texttt{map} function in
    the parser is defined as,

    \begin{lstlisting}[basicstyle=\small]
| MAP LPAR FunArg COMMA Exp RPAR { Map ($3, $5, (), (), $1) }
    \end{lstlisting}

    meaning that the lambda is supposed to passed as a \texttt{FunArg}.  We then
    looked at the declaration of a FunArg,

    \begin{lstlisting}[basicstyle=\small]
FunArg : ID { FunName (#1 $1) },
    \end{lstlisting}

    and added to this definition a case for a lambda function.  In Fasto a
    \texttt{FunArg} is defined as,

    \begin{lstlisting}[basicstyle=\small]
and FunArg = Lambda of Type * Param list * Exp * pos
           | FunName of string.
    \end{lstlisting}

    We then just took the syntax of a lambda and translated that to a list of
    tokens.  Then we pattern matched on that expression and transfered the
    values needed by the lambda in Fasto. Furthermore, Lambda expressions can also
    be called without arguments, therefore we need a case where the Params is non
    existant.

\lstinputlisting[
basicstyle=\footnotesize,
caption={Lambda in Parser.grm },
firstnumber=125,
linerange={125-127},
numbers=left]
{../fasto/src/Parser.grm}

\subsection{Interperter}
    We implemented lambdas in the interpreter by changing the way function
    arguments are evaluated.  We could do this as it is an invariant of the
    Fasto programming language that lambdas can only be used in Second Order
    Array Constructors (SOAC's).  We simply matched a case where
    \texttt{evalFunArg} is called with a lambda instead of a function name.  We
    then use this lambda to construct a function definition with the generic
    name \textit{lambda}.  We then call \texttt{callFunWithVtable} with this
    function declaration and the vtable passed to the function, this means that
    we keep the binding between local variable names and their values and the
    lambda can use these variables.  The function returns an anonymous function
    in sml that takes an argument list and applies the lambda to those
    arguments.

    \begin{lstlisting}[basicstyle=\small]
and evalFunArg (FunName fid, vtab, ftab, callpos) =
    let
      val fexp = SymTab.lookup fid ftab
    in
      case fexp of
        NONE   => raise Error("Function "^fid^" is not in SymTab!", callpos)
      | SOME f => (fn aargs => callFun(f, aargs, ftab, callpos), getFunRTP f)
    end
  | evalFunArg (Lambda (tp, paralist, exp, pos), vtab, ftab, pcall) =
    let val fexp = FunDec ("lambda", tp, paralist, exp, pos)
    in (fn aargs => callFunWithVtable(fexp, aargs, vtab, ftab, pcall), tp)
    end
    \end{lstlisting}

\subsection{Code generator}
The code generation of lambda is a bit complex. The easiest way to explain how it works
is by a list, where each step is followed by the next, in chronological order.
\begin{enumerate}
  \item Remove type decleration from paralist.\\
  We remove the type decleration, beacuse it is not nessesary in the compilation state. This
  prepares us for a relation between the args in the call itself, and the parameters in the
  lambda.
  \item Zip the newly generated list with args.\\
  This assings the args and the parameters in touples. We use a function called zipWith.
  It takes in 2 lists of 2 different types, and a function. Then it iterates over the 2 lists,
  and uses each element of the input lists in the input. Then the function return is stored in
  a new list.
  \item Generate code that assings input from args to the parameter list.\\
  We call zipWith again, but this time we generate a list of mips instructions. In this case
  move the assigned values into the correct registers, so the expression can use these
  \item Generate the local vTable, that contains the local variables.
  \item Compile the expression from the lambda, via compileExp. It takes the local vTable,
  and a register to place its result.
  \item then we concatinate the code from 3 and 5, and returns this.
\end{enumerate}
The source code looks like this
\lstinputlisting[
language=ML,
basicstyle=\footnotesize,
caption={Lambda in CodeGen.sml },
firstnumber=850,
breaklines=true,
linerange={850-865},
numbers=left]
{../fasto/src/CodeGen.sml}



    \section{Task 4 - Copy propagation and constant folding}
	For our implementation of the optimizations copy propagration and constant
	folding, we have made additions to \texttt{CopyConstPropFold.sml}, specifically
	we have added cases for variables, \texttt{let}-bindings, multplication,
	division, logical and/or, logical negation and integer negation in function
	\texttt{copyConstPropFoldExp}.
	\subsection{Multiplication}
	Our implementation of the case for multiplication is based on the simplifications
	given for constant folding of multiplication expressions given in \texttt{GroupProj14.pdf}.
	They are as follows, here \texttt{e1'} and \texttt{e2'} are the recursively optimized
	subexpressions, found in a multiplication expression: $ \texttt{e1} \ast \texttt{e2}$
	\begin{itemize}
	\item if \texttt{e1'} and \texttt{e2'} are constant values then the result will be the
			multiplication value,
	\item if \texttt{e1'} (\texttt{e2'}) is value \texttt{1} then the result is \texttt{e2'}
			(\texttt{e1'}) because $\texttt{1} \ast \texttt{e2'} = \texttt{e1'}$,
	\item if \texttt{e1'} (\texttt{e2'}) is value \texttt{0} then the result is constant value
			\texttt{0} because $\texttt{0} \ast \texttt{e2'} = \texttt{0}$,
	\item otherwise the optimsed result is  \texttt{Times (e1', e2', pos)}.
	\end{itemize}
	Based on this we have implemented our optimization case for multiplication expressions like this:
	\begin{lstlisting}
      | Times (e1, e2, pos) =>
        let val e1' = copyConstPropFoldExp vtable e1
            val e2' = copyConstPropFoldExp vtable e2
        in case (e1', e2') of
               (Constant (IntVal x, _), Constant (IntVal y, _)) =>
               Constant (IntVal (x*y), pos)
             | (Constant (IntVal 0, _), _) =>
               Constant (IntVal 0, pos)
             | (_, Constant (IntVal 0, _)) =>
               Constant (IntVal 0, pos)
             | (Constant (IntVal 1, _), _) =>
               e2'
             | (_, Constant (IntVal 1, _)) =>
               e1'
             | _ =>
               Times (e1', e2', pos)
        end
	\end{lstlisting}
	First the subexpressions, \texttt{e1} and \texttt{e2}, of the multiplication expression are
	optimized, and put respectively in variables \texttt{e1'} and \texttt{e2'}.
	Then the cases described above are run through. A similar rationale have been applied to the
	implementation of the case for division. The implementation can be seen in appendix \ref{CopyConstPropFold}.
	\subsection{Logical and}
	For the case of logical and, we have made the following simplificatons, and based our implementation
	on these. Again \texttt{e1'} and \texttt{e2'} are the recursively optimized subexpressions:
	\begin{itemize}
	\item if \texttt{e1'} and \texttt{e2'} are bools, then the result should be
	\texttt{e1'} \texttt{and} \texttt{e2'}
	\item otherwise the optimsed result is  \texttt{And (e1', e2', pos)}.
	\end{itemize}
	Based on this, we have implemented the case for logical and, in the following way:
	\begin{lstlisting}
      | And (e1, e2, pos) =>
        let val e1' = copyConstPropFoldExp vtable e1
            val e2' = copyConstPropFoldExp vtable e2
        in case (e1', e2') of
               (Constant(BoolVal x,_),Constant(BoolVal y,_)) =>
               Constant(BoolVal (x andalso y),pos)
             | _ =>
               And (e1', e2', pos)
        end
	\end{lstlisting}
	First the subexpressions \texttt{e1} and \texttt{e2}, found in the and-expression, are
	optimized, and put respectively in variables \texttt{e1'} and \texttt{e2'}. The cases descriped above
	are then run through. Our implementation of logical or follows a similar rationale, and can be found in
	appendix \ref{CopyConstPropFold}.
	\subsection{Logical not}
	As with multiplication and logical and, we have made some simplifications describing the
	constant folding on logical not, they are as follows:
	\begin{itemize}
	\item if \texttt{e1'} is true then return false,
	\item if \texttt{e1'} is false then return true,
	\item otherwise the optimsed result is  \texttt{Not (e1', pos)}.
	\end{itemize}
	Our implementation has been based on these simplification, it is found below:
	\begin{lstlisting}
      | Not (e1, pos) =>
        let val e1' = copyConstPropFoldExp vtable e1
        in case (e1') of
               Constant (BoolVal true, _) =>
               Constant (BoolVal false, pos)
             | Constant (BoolVal false, _) =>
               Constant (BoolVal true, pos)
             | _ =>
               Not (e1', pos)
        end
	\end{lstlisting}
	First \texttt{e1} is recursively optimised, and put in \texttt{e1'}. The described cases are
	then run through. Our implementation of integer negation follows a similar rationale, and can
	be found in appenddix \ref{CopyConstPropFold}.
	\subsection{Variables}
	For variables we have implemented the following:
	\begin{lstlisting}
      | Var (name, pos) =>
        let val name = name
            val pos = pos
        in case (SymTab.lookup name vtable) of
            SOME (VarProp newname) => Var (newname, pos)
          | SOME (ConstProp value) => Constant (value, pos)
          | _                      => Var (name, pos)
        end
	\end{lstlisting}
	First the given variable is looked up in the \texttt{vtable}. If a new variable name is returned, our
	variable is thus just a copy of that variable, and we return the new variable, i.e copy propagation. If
	on the other hand a constant value is returned, we return that constant, i.e constant propagation.
	Otherwise our variable is returned as it is, i.e no propagation occurs.
	\subsection{\texttt{let}-bindings}
	We use the term propagatee throughout this part. A propagatee is a variable's defining expression. The
	defining expression of a variable is defined to be either a constant value, or another variable. Below is
	our code for \texttt{let}-bindings. Generally this case detects propagatees in the
	\texttt{let}-binding expression and binds them to the vtable. It also optimizes the expression and the
	body of the \texttt{let}-binding.
	\begin{lstlisting}
      | Let (Dec (name, e, decpos), body, pos) =>
        let val e' = copyConstPropFoldExp vtable e
            val vtable' = bindExpPropagatee name e' vtable
            val body' = copyConstPropFoldExp vtable' body
        in Let (Dec (name, e', decpos), body', pos)
        end
	\end{lstlisting}
	Propagatees are detected and bound to the vtable, by calling function \texttt{bindExpPropagatee} on the optimized expression
	\texttt{e'}and the \texttt{vtable}. \texttt{bindExpPropagatee} calls another function \texttt{expPropagatee}, which extracts
	an propagatee from a given expression. If some propagatee is returned from \texttt{expPropagatee}, it is bound to the vtable.
	\begin{lstlisting}
fun expPropagatee (Var (varname, _)) = SOME (VarProp varname)
  | expPropagatee (Constant (value, _)) = SOME (ConstProp value)
  | expPropagatee _ = NONE

fun bindExpPropagatee name e vtable =
    case expPropagatee e of
        NONE => SymTab.remove name vtable
      | SOME prop => SymTab.bind name prop vtable
	\end{lstlisting}

	\subsection{Tests}
	We have tested out optimization implementations by using the commands: \\
	$\backslash \texttt{bin}\backslash \texttt{fasto -p c testfile.fo}$ and
	$\backslash \texttt{bin}\backslash \texttt{fasto -p testfile.fo}$. With \texttt{-p c}, we only test with the constant folding and
	copy/constant propagation pass, whereas with just \texttt{-p}, we run the only set of optimizations. Our first example is found in \texttt{1optim.fo}, here we first test with \texttt{-p c}.\medskip\\
	\begin{center}
	\begin{minipage}{.5\textwidth}
	\begin{lstlisting}
fun int main() =
    let a = read(int) in
    let b =
        let x = a in
        let y = 2 in
        (x + 2) * (y - 2) in
    b

	\end{lstlisting}
	\end{minipage}%
	\begin{minipage}{.1\textwidth}
	$\rightarrow$
	\end{minipage}%
	\begin{minipage}{.4\textwidth}
	\begin{lstlisting}
fun int main() =
    let a = read(int) in
    let b =
        let x = a in
        let y = 2 in
        0 in
    b
	\end{lstlisting}
	\end{minipage}
	\end{center}
	It is clear that the expression \texttt{(x + 2) $\ast$ (y - 2)} has been optimized to zero, as $\texttt{y = 2}$
	and $\texttt{(y-2) = 0}$. Going further with \texttt{-p}, we get:
	\begin{center}
	\begin{minipage}{.5\textwidth}
	\begin{lstlisting}
fun int main() =
    let a = read(int) in
    let b =
        let x = a in
        let y = 2 in
        (x + 2) * (y - 2) in
    b

	\end{lstlisting}
	\end{minipage}%
	\begin{minipage}{.1\textwidth}
	$\rightarrow$
	\end{minipage}%
	\begin{minipage}{.4\textwidth}
	\begin{lstlisting}
fun int main() =
    let a = read(int) in
    let b = 0 in
    b
	\end{lstlisting}
	\end{minipage}
	\end{center}
	Here the dead variables, \texttt{x} and \texttt{y}, have also been removed. \\
    This refers to \ref{app_title}

    \newpage
    \appendix
    \section{Interpreter AND and OR} \label{interpreter_and_and_or}
    \begin{lstlisting}[basicstyle=\small]
| evalExp ( And(e1, e2, pos), vtab, ftab ) =
      let val r1 = evalExp(e1, vtab, ftab)
      in case r1 of
         BoolVal b1 => if b1 then
                        let val r2 = evalExp(e2, vtab, ftab)
                        in case r2 of
                           BoolVal b2 => BoolVal b2
                         | otherwise  => raise Error ("And expect boolval", pos)
                        end
                       else BoolVal b1
       | otherwise  => raise Error ("And expect boolval", pos)
      end

| evalExp ( Or(e1, e2, pos), vtab, ftab ) =
      let val r1 = evalExp(e1, vtab, ftab)
      in case r1 of
         BoolVal b1 => if not b1 then
                        let val r2 = evalExp(e2, vtab, ftab)
                        in case r2 of
                           BoolVal b2 => BoolVal b2
                         | otherwise  => raise Error ("Or expect boolval", pos)
                        end
                       else BoolVal b1
       | otherwise  => raise Error ("Or expect boolval", pos)
      end
    \end{lstlisting}

    \newpage
    \section{Code Generator AND and OR} \label{code_gen_and_and_or}
    \begin{lstlisting}[basicstyle=\small]
| And (e1, e2, pos) =>
    let val falseLabel = newName "falseLabel"
        val endLabel   = newName "endLabel"
        val t1         = newName "and_L"
        val t2         = newName "and_R"
        val code1      = compileExp e1 vtable t1
        val code2      = compileExp e2 vtable t2
    in code1                            @
       [Mips.BEQ (t1, "0", falseLabel)] @
       code2                            @
       [Mips.BEQ (t2, "0", falseLabel)] @
       [Mips.LI (place, "1")]           @
       [Mips.J endLabel]                @
       [Mips.LABEL falseLabel]          @
       [Mips.LI (place, "0")]           @
       [Mips.LABEL endLabel]
     end
| Or (e1, e2, pos) =>
    let val trueLabel = newName "trueLabel"
        val endLabel  = newName "endLabel"
        val t1        = newName "or_L"
        val t2        = newName "or_R"
        val code1     = compileExp e1 vtable t1
        val code2     = compileExp e2 vtable t2
    in code1                           @
       [Mips.BNE (t1, "0", trueLabel)] @
       code2                           @
       [Mips.BNE (t2, "0", trueLabel)] @
       [Mips.LI (place, "0")]          @
       [Mips.J endLabel]               @
       [Mips.LABEL trueLabel]          @
       [Mips.LI (place, "1")]          @
       [Mips.LABEL endLabel]
    end
    \end{lstlisting}

    \newpage
    \section{Interpreter NOT} \label{interpreter_not}
    \begin{lstlisting}[basicstyle=\small]
| evalExp ( Not(e1, pos), vtab, ftab ) =
      let val r1 = evalExp(e1, vtab, ftab)
      in case r1 of
          BoolVal true  => BoolVal false
        | BoolVal false => BoolVal true
        | other         => raise Error("Not expects a boolean value", pos)
      end
    \end{lstlisting}

    \newpage
    \section{Code Generation NOT} \label{code_gen_not}
    \begin{lstlisting}[basicstyle=\small]
| Not (e1, pos) =>
    let val zeroLabel = newName "zeroLabel"
        val endLabel  = newName "endLabel"
        val t1        = newName "not_R"
        val code      = compileExp e1 vtable t1
    in code                            @
       [Mips.BEQ (t1, "0", zeroLabel)] @
       [Mips.XOR (place, t1, t1)]      @
       [Mips.J endLabel]               @
       [Mips.LABEL zeroLabel]          @
       [Mips.LI (place, "1")]          @
       [Mips.LABEL endLabel]
    end
    \end{lstlisting}

    \newpage
    \section{Boolean Implementation} \label{boolean_implementation}
    \subsection{Lexer}
    \begin{lstlisting}[basicstyle=\small]
| "true"         => Parser.BOOLLIT (true, pos)
| "false"        => Parser.BOOLLIT (false, pos)
    \end{lstlisting}

    \subsection{Parser}
    \begin{lstlisting}[basicstyle=\small]
(...)
%token <bool*(int*int)> BOOLLIT
(...)
| BOOLLIT             { Constant (BoolVal (#1 $1), #2 $1) }
    \end{lstlisting}

    \subsection{Interpreter}
    \begin{lstlisting}[basicstyle=\small]
fun evalExp ( Constant (v,_), vtab, ftab ) = v
    \end{lstlisting}

    \subsection{Typechecker}
    \begin{lstlisting}[basicstyle=\small]
In.Constant  (v, pos)     => (valueType v, Out.Constant (v, pos))
    \end{lstlisting}

    \subsection{Code Generation}
    \begin{lstlisting}[basicstyle=\small]
| Constant (BoolVal b, pos) => if b
                               then [Mips.LI (place, "1")]
                               else [Mips.LI (place, "0")]
    \end{lstlisting}

    \newpage
    \section{Task 1 Tests} \label{task_1_tests}
    \subsection{and\_err.fo}
    \begin{lstlisting}[basicstyle=\small]
fun bool main() =
    true && '5'
    \end{lstlisting}

    \subsection{andOr.fo}
    \begin{lstlisting}[basicstyle=\small]
fun [char] writebool(bool b) = if b then write("true") else write("false")

fun int main() =
    let a = 1 == 1 in        // a = true
    let b = 1 == 2 in        // a = false
    let c = a || b && a in   // c = true
    let d = a && b in        // d = false
    let e = a || b in        // e = true
    let f = (a || b) && b in // f = false
    let w = writebool(a) in
    let w = writebool(b) in
    let w = writebool(c) in
    let w = writebool(d) in
    let w = writebool(e) in
    let w = writebool(f) in
        0
    \end{lstlisting}

    \subsection{and\_sc.fo}
    \begin{lstlisting}[basicstyle=\small]
fun int main() =
    let a = {true, false, true, true, true, false} in
        if false && a[100] // Array index out of bounds
        then write(1)
        else write(0)
    \end{lstlisting}

    \subsection{boolCompare.fo}
    \begin{lstlisting}[basicstyle=\small]
fun [char] printbool(bool b) = if b then write("true ") else write("false ")

fun int main() =
    let False = false in
    let True = true in
    let w = printbool(False) in // false
    let w = printbool(True) in // true
    let w = printbool(True == True) in // true
    let w = printbool(True == False) in // false
    let w = printbool(True == False || not False == True) in // true
    let w = printbool(1+2 == 2) in // false
    let w = printbool(10*2 == 20 ) in // true
    let w = printbool(10/2 == 5 ) in // true
    let w = printbool(False == True || not False && True == false) in // false
        0
    \end{lstlisting}

    \subsection{boolLit.fo}
    \begin{lstlisting}[basicstyle=\small]
fun [char] writebool(bool b) = if b then write("true") else write("false")

fun int main() =
    let a = true in
    let b = false in
    let w = writebool(a) in
    let w = writebool(b) in
        0
    \end{lstlisting}

    \subsection{div\_err.fo}
    \begin{lstlisting}[basicstyle=\small]
fun int main() =
    4 / 'h'
    \end{lstlisting}

    \subsection{intNegate.fo}
    \begin{lstlisting}[basicstyle=\small]
fun int test(int x) =
    let w = write (x)  in
    0

fun int main() =
    let w =  test(3/2) in
    let w =  test(5/3) in
    let w =  test(10/9) in
    let w =  test(10/11) in
    let w =  test(~10/2) in
    let w =  test(~10/~2) in
    let w =  test(10/~2) in
    let w =  test(5/~3) in
    let w =  test(1+2+5/3*9+3) in
    let w =  test(5+10/~2*3-~2) in //(5+((5/(-2))*3))-(-3)= -1.5 ~~ -1

    0
    \end{lstlisting}

    \subsection{muldivide.fo}
    \begin{lstlisting}[basicstyle=\small]
fun int main() =
    let q = 3*5 in
    let w = 4/2 in
    let r = 5/2 in
    let g = write(q) in
    let g = write(w) in
    let g = write(r) in
    0
    \end{lstlisting}

    \subsection{mul\_err.fo}
    \begin{lstlisting}[basicstyle=\small]
fun int main() =
    4 * 'h'
    \end{lstlisting}

    \subsection{negate2.fo}
    \begin{lstlisting}[basicstyle=\small]
fun int main() =
    let w = write(~2 * 4) in
    let w = write(2 * 4) in
    let w = write(~2 - ~2) in
    let w = write(~5 + 5) in
        0
    \end{lstlisting}

    \subsection{negate.fo}
    \begin{lstlisting}[basicstyle=\small]
fun bool main() =
    let x0 =  write(3 / 2 ==  1) in
    let x1 = write(~3 / 2 == ~2) in
    let x2 =  write(3 /~2 == ~2) in
    let x3 = write(~3 /~2 ==  1) in
    let x4 = write(~(5+5) ==  ~10) in
    let x5 = write(~(5+5)) in
    let x5 = write(~(1)) in
    write(x0 && x1 && x2 && x3 && x4)
    \end{lstlisting}

    \subsection{neg\_err.fo}
    \begin{lstlisting}[basicstyle=\small]
fun int main() =
    let a = 5 in
    let b = 'h' in
    ~a + ~b
    \end{lstlisting}

    \subsection{not\_err.fo}
    \begin{lstlisting}[basicstyle=\small]
fun bool main() =
    if not 'h' then true else false
    \end{lstlisting}

    \subsection{or\_err.fo}
    \begin{lstlisting}[basicstyle=\small]
fun bool main() =
    false || false || {'h', 'e', 'j'}
    \end{lstlisting}

    \subsection{or\_sc.fo}
    \begin{lstlisting}[basicstyle=\small]
fun int main() =
    let a = {true, true, false, true} in
        if true || a[100] // Array index out of bounds
        then 0
        else 1
    \end{lstlisting}

    \subsection{tobeornottobe.fo}
    \begin{lstlisting}[basicstyle=\small]
fun [char] printbool(bool b) = if b then write("true ") else write("false ")

fun int main() =
    let to = false in
    let be = false in
    let w = printbool(to == be || not to && be == true) in // true
    let to = false in
    let be = true in
    let w = printbool(to == be || not to && be == true) in // true
    let to = true in
    let be = false in
    let w = printbool(to == be || not to && be == true) in // false
    let to = true in
    let be = true in
    let w = printbool(to == be || not to && be == true) in // true
        0
    \end{lstlisting}

    \newpage
    \section{Interpreter Scan} \label{interpreter_scan}
    \begin{lstlisting}[basicstyle=\small]
| In.Scan (f, n_exp, arr_exp, _, pos)
  => let val (n_type, n_dec) = checkExp ftab vtab n_exp
         val (arr_type, arr_dec) = checkExp ftab vtab arr_exp
         val elem_type =
           case arr_type of
               Array t => t
             | other => raise Error ("Scan: Argument not an array", pos)
         val (f', f_arg_type) =
           case checkFunArg (f, vtab, ftab, pos) of
               (f', res, [a1, a2]) =>
               if a1 = a2 andalso a2 = res
               then (f', res)
               else raise Error
                      ("Scan: incompatible function type of "
                       ^ In.ppFunArg 0 f ^": " ^ showFunType ([a1, a2], res), pos)
             | (_, res, args) =>
               raise Error ("Scan: incompatible function type of "
                            ^ In.ppFunArg 0 f ^ ": " ^ showFunType (args, res), pos)
         fun err (s, t) =
             Error ("Scan: unexpected " ^ s ^ " type " ^ ppType t ^
                    ", expected " ^ ppType f_arg_type, pos)
     in if elem_type = f_arg_type
        then if elem_type = n_type
             then (arr_type,
                   Out.Scan (f', n_dec, arr_dec, elem_type, pos))
             else raise err ("neutral element", n_type)
        else raise err ("array element", elem_type)
     end
    \end{lstlisting}

    \newpage
    \section{Code Generation Filter and Scan} \label{code_gen_scan_filter}
    \begin{lstlisting}[basicstyle=\small]
| Scan (farg, acc_exp, arr_exp, tp, pos) =>
  let val arr_reg   = newName "arr_reg"   (* Address of array given. *)
      val size_reg  = newName "size_reg"  (* Size of original array. *)
      val i_reg     = newName "i_reg"     (* Counter register. *)
      val new_size  = newName "new_size"  (* Size of new array. *)
      val tmp_reg   = newName "tmp_reg"   (* Various uses. *)
      val res_reg   = newName "res_reg"   (* Various uses. *)
      val startlabl = newName "startlabl" (* Start label of while loop. *)
      val endlabl   = newName "endlabl"   (* The end of while loop. *)
      val neu_el    = newName "neu_el"    (* Neutral element. *)
      val elreadreg = newName "elreadreg" (* Element to be read. *)
      val elwritreg = newName "elwritreg" (* Where to write. *)

      val arr_code  = compileExp arr_exp vtable arr_reg
      val get_size  = [ Mips.LW (size_reg, arr_reg, "0") ]
      val comp_size = [ Mips.ADDI (new_size, size_reg, "1") ]
      val neu_code  = compileExp acc_exp vtable neu_el

      val set_r_w   = case getElemSize tp of
                        One  => [ Mips.ADDI (elreadreg, arr_reg, "4")
                                , Mips.ADDI (elwritreg, place, "5") ]
                      | Four => [ Mips.ADDI (elreadreg, arr_reg, "4")
                                , Mips.ADDI (elwritreg, place, "8") ]

      val whileinit = case getElemSize tp of
                        One  => [ Mips.LI (i_reg, "0")
                                , Mips.SB (neu_el, place, "4") ]
                      | Four => [ Mips.LI (i_reg, "0")
                                , Mips.SW (neu_el, place, "4") ]

      val whilecond = [ Mips.LABEL startlabl
                      , Mips.SLT (tmp_reg, i_reg, size_reg)
                      , Mips.BEQ (tmp_reg, "0", endlabl) ]

      val whileloop = case getElemSize tp of
                        One  => [ Mips.LB (res_reg, elreadreg, "0")
                                , Mips.LB (tmp_reg, elwritreg, "-1") ]
                                @ applyFunArg(farg, [tmp_reg, res_reg],
                                              vtable, res_reg, pos)
                                @ [ Mips.SB (res_reg, elwritreg, "0")
                                , Mips.ADDI (elreadreg, elreadreg, "1")
                                , Mips.ADDI (elwritreg, elwritreg, "1")
                                , Mips.ADDI (i_reg, i_reg, "1")
                                , Mips.J startlabl ]
                      | Four => [ Mips.LW (res_reg, elreadreg, "0")
                                , Mips.LW (tmp_reg, elwritreg, "-4") ]
                                @ applyFunArg(farg, [tmp_reg, res_reg],
                                              vtable, res_reg, pos)
                                @ [ Mips.SW (res_reg, elwritreg, "0")
                                , Mips.ADDI (elreadreg, elreadreg, "4")
                                , Mips.ADDI (elwritreg, elwritreg, "4")
                                , Mips.ADDI (i_reg, i_reg, "1")
                                , Mips.J startlabl ]

      val while_end = [ Mips.LABEL endlabl ]

  in arr_code  @ (* Compute input array expr. *)
     get_size  @ (* Save size of input array to size_reg. *)
     comp_size @ (* Calculate size of new arr. *)
     neu_code  @ (* Compute the neutral element. *)
     dynalloc (new_size, place, tp) @ (* Allocates space for res. *)
     set_r_w   @ (* Set read and write registers. *)
     whileinit @ (* While initialization. *)
     whilecond @ (* Test the while condition. *)
     whileloop @ (* Code of while. *)
     while_end   (* While end label. *)
  end

| Filter (farg, arr_exp, elem_type, pos) =>
  let val size_reg = newName "size_reg" (* size of input/output array *)
      val arr_reg  = newName "arr_reg" (* address of array *)
      val elem_reg = newName "elem_reg" (* address of single element *)
      val res_reg = newName "res_reg"
      val arr_code = compileExp arr_exp vtable arr_reg

      val get_size = [ Mips.LW (size_reg, arr_reg, "0") ]

      val addr_reg = newName "addr_reg" (* address of element in new array *)
      val i_reg = newName "i_reg" (*amount of progress of iteration in input array*)
      val j_reg = newName "j_reg" (*Count of true elements in output*)
      val init_regs = [ Mips.ADDI (addr_reg, place, "4")
                      , Mips.MOVE (i_reg, "0")
                      , Mips.MOVE (j_reg, "0")
                      , Mips.ADDI (elem_reg, arr_reg, "4") ]

      val loop_beg = newName "loop_beg"
      val loop_end = newName "loop_end"
      val value_reg = newName "value_reg"
      val false_= newName "false_"
      val tmp_reg = newName "tmp_reg"
      val loop_header = [ Mips.LABEL (loop_beg)
                        , Mips.SUB (tmp_reg, i_reg, size_reg)
                        , Mips.BGEZ (tmp_reg, loop_end) ]

      (* map is 'arr[i] = f(old_arr[i])'. *)
      val loop_filter0 =
          case getElemSize elem_type of
              One => Mips.LB(res_reg, elem_reg, "0")
                     :: applyFunArg(farg, [res_reg], vtable, res_reg, pos)
            | Four => Mips.LW(res_reg, elem_reg, "0")
                      :: applyFunArg(farg, [res_reg], vtable, res_reg, pos)

      val loop_filter1 =
          case getElemSize elem_type of
              One =>  [  Mips.BEQ(res_reg, "0" ,false_)
                       , Mips.LB (value_reg, elem_reg, "0")
                       , Mips.SB (value_reg, addr_reg, "0")]
            | Four => [  Mips.BEQ(res_reg, "0" ,false_)
                       , Mips.LW (value_reg, elem_reg, "0")
                       , Mips.SW (value_reg, addr_reg, "0")]

      val loop_footer =
          [ Mips.ADDI (addr_reg, addr_reg,
                       makeConst (elemSizeToInt (getElemSize elem_type)))
          , Mips.ADDI (j_reg, j_reg, "1")
          , Mips.LABEL false_
          , Mips.ADDI(elem_reg, elem_reg,
                       makeConst (elemSizeToInt (getElemSize elem_type)))
          , Mips.ADDI (i_reg, i_reg, "1")
          , Mips.J loop_beg
          , Mips.LABEL loop_end
          , Mips.SW (j_reg, place, "0")
          ]
  in arr_code
     @ get_size
     @ dynalloc (size_reg, place, elem_type)
     @ init_regs
     @ loop_header
     @ loop_filter0
     @ loop_filter1
     @ loop_footer
  end
    \end{lstlisting}

\end{document}
