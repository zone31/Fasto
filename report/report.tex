\documentclass[11pt]{article}
\usepackage[a4paper, hmargin={2.8cm, 2.8cm}, vmargin={2.5cm, 2.5cm}]{geometry}
\usepackage{eso-pic} % \AddToShipoutPicture
\usepackage{graphicx} % \includegraphics
\usepackage{fancyhdr, amsmath, amssymb, comment, caption, placeins, subfigure,
    fixltx2e, changepage, listings, courier, soul, hyperref, geometry,
    enumerate, dsfont, listings}

\author{
    \Large{Magnus N\o rskov Stavngaard, Mark Jan Jacobi and Christian
        Salb\ae k} \\
    \texttt{magnus@stavngaard.dk, mark@jacobi.pm and chr.salbaek@gmail.com} \\
}

\title{
    \vspace{3cm}
    \Huge{Compiler Design} \\
    \Large{Compiler for the Fasto Programming Language}
}

\pagestyle{fancy}
\lhead{Magnus N. S. Mark J. J. Christian S.}
\chead{\date{\today}}
\rhead{University of Copenhagen}
% \lfoot{}
% \cfoot{}
% \rfoot{}

% Change indent length of paragraph not after a header.
\setlength{\parindent}{0cm}

% Remove page numbering in the beginning
\pagenumbering{gobble}

\begin{document}

    %% Change `ku-farve` to `nat-farve` to use SCIENCE's old colors or
    %% `natbio-farve` to use SCIENCE's new colors and logo.
    \AddToShipoutPicture*{\put(0,0){\includegraphics*[viewport=0 0 700 600]
        {include/ku-farve}}}
    \AddToShipoutPicture*{\put(0,602){\includegraphics*[viewport=0 600 700 1600]
        {include/ku-farve}}}

    %% Change `ku-en` to `nat-en` to use the `Faculty of Science` header
    \AddToShipoutPicture*{\put(0,0){\includegraphics*{include/ku-en}}}

    \clearpage
    \maketitle
    \thispagestyle{empty}

    \newpage

    \pagenumbering{arabic} % Arabic page numbers (and reset to 1)

    \section{Task 1 - Warmup}
    In task 1 we were asked to implement the boolean operators \texttt{\&\&},
    \texttt{||} and \texttt{not}, the boolean constants true and false, integer
    multiplication, integer division and integer negation. \\

    We will go through in detail the implementation of integer division and
    multiplication and then skip rather quickly over the implementation of the
    rest of the operators only describing what is different from multiplication
    and division as the operations is implemented very similar.

    \subsection{Integer Multiplication and Division}
    We started by implementing integer multiplication and division in the Lexer.
    We created a new rule for the star and division operator that created tokens
    and passed the tokens to the parser.

    \begin{lstlisting}[basicstyle=\small]
| `*`   { Parser.TIMES    (getPos lexbuf) }
| `/`   { Parser.DIVIDE   (getPos lexbuf) }
    \end{lstlisting}

    In the parser we added the tokens where addition and subtraction was already
    defined, as integer multiplication and division has allot in common with
    addition and subtraction.  Integer multiplication and division carries two
    integers corresponding to a position in the code.

    \begin{lstlisting}[basicstyle=\small]
%token <(int*int)> PLUS MINUS TIMES DIVIDE DEQ EQ LTH BOOLAND BOOLOR NOT NEG
    \end{lstlisting}

    We also declare both times and divide as left associative operators with
    greater precedence than addition and subtraction.

    \begin{lstlisting}[basicstyle=\small]
%left BOOLOR
%left BOOLAND
%left NOT
%left DEQ LTH
%left PLUS MINUS
%left TIMES
%left DIVIDE
%left NEG
    \end{lstlisting}

    We then defined that an expression could consist of an expression followed
    by a multiplication or division followed by an expression.  And that this
    correspond to \texttt{Times} and \texttt{Divide} in the Fasto language definition.

    \begin{lstlisting}[basicstyle=\small]
Exp :     NUM                 { Constant (IntVal (#1 $1), #2 $1) }
        | CHARLIT             { Constant (CharVal (#1 $1), #2 $1) }

        (...)

        | Exp TIMES Exp       { Times($1, $3, $2) }
        | Exp DIVIDE Exp      { Divide($1, $3, $2) }

        (...)
    \end{lstlisting}

    In the interpreter we implemented cases for \texttt{Times} and
    \texttt{Divide} in the \texttt{evalExpr} function.

    \begin{lstlisting}[basicstyle=\small]
| evalExp ( Times(e1, e2, pos), vtab, ftab ) =
      let val res1 = evalExp(e1, vtab, ftab)
          val res2 = evalExp(e2, vtab, ftab)
      in  evalBinopNum(op *, res1, res2, pos)
      end

| evalExp ( Divide(e1, e2, pos), vtab, ftab ) =
      let val res1 = evalExp(e1, vtab, ftab)
          val res2 = evalExp(e2, vtab, ftab)
      in  evalBinopNum(op Int.quot, res1, res2, pos)
      end
    \end{lstlisting}

    Our cases evaluate recursively the expressions to the left and right of the
    operator and then calls \texttt{evalBinopNum} with the appropriate operator
    and the results from evaluating the lefthand and the righthand side of the
    expression. \\

    We then implemented the operators in the typechecker.  Our cases call a
    helper function \texttt{checkBinOp} that takes a position, an expected type,
    and two expressions and check that the two expressions have the type of the
    expected type.  If the types match the types is returned with
    \textit{typedecorated} versions of the expressions, if the types doesn't
    match an error is raised. \\
    We then simply return the same operation, now with a return type.

    \begin{lstlisting}[basicstyle=\small]
| In.Times (e1, e2, pos)
  => let val (_, e1_dec, e2_dec) = checkBinOp ftab vtab (pos, Int, e1, e2)
     in (Int, Out.Times (e1_dec, e2_dec, pos))
     end

| In.Divide (e1, e2, pos)
  => let val (_, e1_dec, e2_dec) = checkBinOp ftab vtab (pos, Int, e1, e2)
     in (Int, Out.Divide (e1_dec, e2_dec, pos))
     end
    \end{lstlisting}

    We can now finally implement the operators in the code generator.  Here we
    create two temporary variables \texttt{t1} and \texttt{t2} to hold the
    values of the expression on either side of the operator.  We then call the
    function \texttt{compileExp} recursively with these names to get the machine
    code for the expression on either side of the operator.  Then we just simply
    return a list of first the code to compute the left hand side of the
    operator, then the right hand side, and then we apply the MIPS commands
    \texttt{MUL} and \texttt{DIV} to the two \textit{subresults} and save the
    result in \texttt{place}.

    \begin{lstlisting}[basicstyle=\small]
| Times (e1, e2, pos) =>
    let val t1 = newName "times_L"
        val t2 = newName "times_R"
        val code1 = compileExp e1 vtable t1
        val code2 = compileExp e2 vtable t2
    in  code1 @ code2 @ [Mips.MUL (place, t1, t2)]
    end
| Divide (e1, e2, pos) =>
    let val t1 = newName "divide_L"
        val t2 = newName "divide_R"
        val code1 = compileExp e1 vtable t1
        val code2 = compileExp e2 vtable t2
    in  code1 @ code2 @ [Mips.DIV (place, t1, t2)]
    end
    \end{lstlisting}

\end{document}
