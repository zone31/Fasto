\documentclass[11pt]{article}
\usepackage[a4paper, hmargin={2.8cm, 2.8cm}, vmargin={2.5cm, 2.5cm}]{geometry}
\usepackage{eso-pic} % \AddToShipoutPicture
\usepackage{graphicx} % \includegraphics
\usepackage{fancyhdr, amsmath, amssymb, comment, caption, placeins, subfigure,
    fixltx2e, changepage, listings, courier, soul, hyperref, geometry,
    enumerate, dsfont, listings, enumitem}
    \usepackage[T1]{fontenc}





\lstset{
basicstyle=\footnotesize,
breaklines=false,
language=ML,
numbers=left}




\author{\Large{Magnus N\o rskov Stavngaard} \\
		\texttt{magnus@stavngaard.dk}
		\\\\
		\Large{Mark Jan Jacobi} \\
        \texttt{mark@jacobi.pm}
		 \\\\
		\Large{Christian Salb\ae k} \\
		\texttt{chr.salbaek@gmail.com}
}

\title{
    \vspace{3cm}
    \Huge{Compiler Design} \\
    \Large{Compiler for the Fasto Programming Language}
}

\pagestyle{fancy}
\lhead{\small{Magnus N. S. Mark J. J. Christian S.}}
\chead{\date{\today}}
\rhead{University of Copenhagen}
% \lfoot{}
% \cfoot{}
% \rfoot{}

% Change indent length of paragraph not after a header.
\setlength{\parindent}{0cm}

% Remove page numbering in the beginning
\pagenumbering{gobble}

\begin{document}

    %% Change `ku-farve` to `nat-farve` to use SCIENCE's old colors or
    %% `natbio-farve` to use SCIENCE's new colors and logo.
    \AddToShipoutPicture*{\put(0,0){\includegraphics*[viewport=0 0 700 600]
        {include/ku-farve}}}
    \AddToShipoutPicture*{\put(0,602){\includegraphics*[viewport=0 600 700 1600]
        {include/ku-farve}}}

    %% Change `ku-en` to `nat-en` to use the `Faculty of Science` header
    \AddToShipoutPicture*{\put(0,0){\includegraphics*{include/ku-en}}}

    \clearpage
    \maketitle
    \thispagestyle{empty}

    \newpage

    \tableofcontents

    \newpage

    \pagenumbering{arabic} % Arabic page numbers (and reset to 1)

    \section{Task 1 - Warmup}
    In task 1 we were asked to implement the boolean operators \texttt{\&\&},
    \texttt{||} and \texttt{not}, the boolean constants true and false, integer
    multiplication, integer division and integer negation. \\

    We will go through in detail the implementation of integer division and
    multiplication and then skip rather quickly over the implementation of the
    rest of the operators only describing what is different from multiplication
    and division as the operations is implemented very similar.

    \subsection{Integer Multiplication and Division}
    We started by implementing integer multiplication and division in the Lexer.
    We created a new rule for the star and division operator that created tokens
    and passed the tokens to the parser.

    \begin{lstlisting}[basicstyle=\small]
| `*`   { Parser.TIMES    (getPos lexbuf) }
| `/`   { Parser.DIVIDE   (getPos lexbuf) }
    \end{lstlisting}

    In the parser we added the tokens where addition and subtraction was already
    defined, as integer multiplication and division has allot in common with
    addition and subtraction.  Integer multiplication and division carries two
    integers corresponding to a position in the code.

    \begin{lstlisting}[basicstyle=\small]
%token <(int*int)> PLUS MINUS TIMES DIVIDE DEQ EQ LTH BOOLAND BOOLOR NOT NEG
    \end{lstlisting}

    We also declare both times and divide as left associative operators with
    greater precedence than addition and subtraction.

    \begin{lstlisting}[basicstyle=\small]
%left BOOLOR
%left BOOLAND
%left NOT
%left DEQ LTH
%left PLUS MINUS
%left TIMES DIVIDE
%left NEG
    \end{lstlisting}

    We then defined that an expression could consist of an expression followed
    by a multiplication or division followed by an expression.  And that this
    correspond to \texttt{Times} and \texttt{Divide} in the Fasto language
    definition.

    \begin{lstlisting}[basicstyle=\small]
Exp :     NUM                 { Constant (IntVal (#1 $1), #2 $1) }
        | CHARLIT             { Constant (CharVal (#1 $1), #2 $1) }

        (...)

        | Exp TIMES Exp       { Times($1, $3, $2) }
        | Exp DIVIDE Exp      { Divide($1, $3, $2) }

        (...)
    \end{lstlisting}

    In the interpreter we implemented cases for \texttt{Times} and
    \texttt{Divide} in the \texttt{evalExpr} function.

    \begin{lstlisting}[basicstyle=\small]
| evalExp ( Times(e1, e2, pos), vtab, ftab ) =
      let val res1 = evalExp(e1, vtab, ftab)
          val res2 = evalExp(e2, vtab, ftab)
      in  evalBinopNum(op *, res1, res2, pos)
      end

| evalExp ( Divide(e1, e2, pos), vtab, ftab ) =
      let val res1 = evalExp(e1, vtab, ftab)
          val res2 = evalExp(e2, vtab, ftab)
      in  evalBinopNum(op Int.quot, res1, res2, pos)
      end
    \end{lstlisting}

    Our cases evaluate recursively the expressions to the left and right of the
    operator and then calls \texttt{evalBinopNum} with the appropriate operator
    and the results from evaluating the lefthand and the righthand side of the
    expression. \\

    We then implemented the operators in the typechecker.  Our cases call a
    helper function \texttt{checkBinOp} that takes a position, an expected type,
    and two expressions and check that the two expressions have the type of the
    expected type.  If the types match the types is returned with
    \textit{typedecorated} versions of the expressions, if the types doesn't
    match an error is raised. \\
    We then simply return the same operation, now with a return type.

    \begin{lstlisting}[basicstyle=\small]
| In.Times (e1, e2, pos)
  => let val (_, e1_dec, e2_dec) = checkBinOp ftab vtab (pos, Int, e1, e2)
     in (Int, Out.Times (e1_dec, e2_dec, pos))
     end

| In.Divide (e1, e2, pos)
  => let val (_, e1_dec, e2_dec) = checkBinOp ftab vtab (pos, Int, e1, e2)
     in (Int, Out.Divide (e1_dec, e2_dec, pos))
     end
    \end{lstlisting}

    We can now finally implement the operators in the code generator.  Here we
    create two temporary variables \texttt{t1} and \texttt{t2} to hold the
    values of the expression on either side of the operator.  We then call the
    function \texttt{compileExp} recursively with these names to get the machine
    code for the expression on either side of the operator.  Then we just simply
    return a list of first the code to compute the left hand side of the
    operator, then the right hand side, and then we apply the MIPS commands
    \texttt{MUL} and \texttt{DIV} to the two \textit{subresults} and save the
    result in \texttt{place}.

    \begin{lstlisting}[basicstyle=\small]
| Times (e1, e2, pos) =>
    let val t1 = newName "times_L"
        val t2 = newName "times_R"
        val code1 = compileExp e1 vtable t1
        val code2 = compileExp e2 vtable t2
    in  code1 @ code2 @ [Mips.MUL (place, t1, t2)]
    end
| Divide (e1, e2, pos) =>
    let val t1 = newName "divide_L"
        val t2 = newName "divide_R"
        val code1 = compileExp e1 vtable t1
        val code2 = compileExp e2 vtable t2
    in  code1 @ code2 @ [Mips.DIV (place, t1, t2)]
    end
    \end{lstlisting}

    \subsection{Boolean Operators}
    We started by implementing the boolean operators in the lexer in a very
    similar way that we implemented multiplication and division.  In the parser,
    we made sure that the boolean operators were defined as having a lower
    precedence than the arithmetic operators, so that an expression like,

    \begin{lstlisting}[basicstyle=\small]
        2 + 4 == 6 || 5 + 8 == 10 && 8 < 10
    \end{lstlisting}

    is evaluated like,

    \begin{lstlisting}[basicstyle=\small]
        ((2 + 4) == 6) || (((5 + 8) == 10) && (8 < 10)).
    \end{lstlisting}

    Notice that the \texttt{\&\&} operator is also evaluated before the
    \texttt{||} operator. \\

    After this we implemented the operators in the interpreter.  Here we created
    a case for \texttt{and} and a case for \texttt{or}.  We had to implement
    them as short circuited which means that the right hand side of an
    \texttt{and} should only be evaluated if the left hand side is true.
    Similarly for \texttt{or} the right hand side should only be evaluated if
    the left hand side is false (The or implementation can be seen in appendix
    \ref{interpreter_and_and_or}.

    \begin{lstlisting}[basicstyle=\small]
| evalExp ( And(e1, e2, pos), vtab, ftab ) =
      let val r1 = evalExp(e1, vtab, ftab)
      in case r1 of
         BoolVal b1 => if b1 then
                        let val r2 = evalExp(e2, vtab, ftab)
                        in case r2 of
                           BoolVal b2 => BoolVal b2
                         | otherwise  => raise Error ("And expect boolval", pos)
                        end
                       else BoolVal b1
       | otherwise  => raise Error ("And expect boolval", pos)
      end

    \end{lstlisting}

    We first evaluate the lefthand expression, if that is true we evaluate the
    right hand expression, if that is also true we return True otherwise we
    return false.  If either of the expressions isn't a BoolVal, we report an
    error. \\

    In the code generator it was also required that we implemented the boolean
    operators to be short-circuiting so that the right hand side of an
    \texttt{and} only evaluates if the left hand side is true.  Similarly the
    right hand side of an \texttt{or} evaluates only if the left hand side is
    false.  We did this with the MIPS assembly code,

    \begin{lstlisting}[basicstyle=\small]
$t1 = compile e1

beq $t1, $zero, False

$t2 = compile e2

beq $t2, $zero, False
li  $s1, 1 ;; Assuming the result should be saved to $s1
j   End

False:
    li  $s1, 0 ;; Assuming the result should be saved to $s2

End:
    \end{lstlisting}

    It can be seen that if the first part of an \texttt{and} return false i.e.
    the result register contains 0, we simply skip over the execution of the
    right hand side and jump strait to False.  We did something similar for
    \texttt{or}'s.  The Standard ML code generating the MIPS assembly for both
    \texttt{and} and \texttt{or} can be seen in appendix
    \ref{code_gen_and_and_or}.

    \subsection{Boolean Negation}
    In the lexer we implemented the operator for boolean negation \texttt{not}
    as a keyword.  We did this because \texttt{not} is a valid variable name and
    would pass the rule,

    \begin{lstlisting}[basicstyle=\small]
| [`a`-`z` `A`-`Z`] [`a`-`z` `A`-`Z` `0`-`9` `_`]*
                      { keyword (getLexeme lexbuf,getPos lexbuf) }.
    \end{lstlisting}

    If we didn't implement a keyword, \texttt{not} would simply fall through and
    go in the case,

    \begin{lstlisting}
| _              => Parser.ID (s, pos)
    \end{lstlisting}

    in the \texttt{keyword} function.  The implemented keyword goes to a
    \texttt{Parser.NOT} in the parser and carries only the position.

    \begin{lstlisting}[basicstyle=\small]
fun keyword (s, pos) =
    case s of
         "if"           => Parser.IF pos
       | "then"         => Parser.THEN pos

       (...)

       | "not"          => Parser.NOT pos
       | "fn"           => Parser.LAMBDA pos
       | _              => Parser.ID (s, pos)
}
    \end{lstlisting}

    In the interpreter \texttt{not} is implemented simply by evaluating the
    expression after the \texttt{not}.  If that expression results in a true,
    false is returned, if it results in a false, true is returned and otherwise
    an error is reported.  The code can be seen in appendix
    \ref{interpreter_not}. \\

    \texttt{not} is implemented in the code generator as a branch operation.  If
    \texttt{not} is applied to true 3 operations are performed, if it is applied
    to false only 2 operations are performed.  The code generating MIPS assembly
    can be found in appendix \ref{code_gen_not}.

    \subsection{Integer Negation}
    Integer negation has been implemented two places in the lexer.  For integer
    constants which is negated the negated value is simply created on
    compiletime.  This is done by the rule,

    \begin{lstlisting}[basicstyle=\small]
| [`0`-`9`]+ | "~" [`0`-`9`]+ { case Int.fromString (getLexeme lexbuf) of
                             NONE   => lexerError lexbuf "Bad integer"
                           | SOME i => Parser.NUM (i, getPos lexbuf) }
    \end{lstlisting}

    which says that a number or a tilde followed by a one or more numbers is a
    Parser.NUM in the parser and the integer value is carried with it. \\

    If a tilde is not followed by a number for example in,

    \begin{lstlisting}
~(2 + 4) - ~(2 - ~f(3))
    \end{lstlisting}

    the negation is caught by the rule,

    \begin{lstlisting}
| "~"                 { Parser.NEG      (getPos lexbuf) }
    \end{lstlisting}

    In the code generator the negations is performed as a exclusive or and an
    addition.

    \begin{lstlisting}[basicstyle=\small]
| Negate (e, pos) =>
    let val negThis  = newName "negThis"
        val code     = compileExp e vtable negThis
        val negation =
            [Mips.XORI(negThis,negThis,"-1")] @ [Mips.ADDI (place,negThis,"1")]
    in code @ negation
    end
    \end{lstlisting}

    This works because computers use two's complement to express numbers.  In
    two's complement the negation of a number is computed by flipping all bits
    in the number and adding 1.  We flip all bits by exclusive or'ring with -1,
    the binary value of -1 is 1111 1111.  Therefore all the places where there
    was 0 in the original number there will now be 1 and where there were 1
    there will now be 0.  After that we simply add 1 with \texttt{addi} and save
    the result to place.

    \subsection{Boolean Literals}
    The boolean literals is implemented in the lexer as a keyword that carries a
    boolean value to the parser.  The implementation of boolean literals in all
    compiler phases can be found in appendix \ref{boolean_implementation}.

    \subsection{Test}
    \begin{tabular}{|l|l|}
        \hline
        \textbf{Test name} & \textbf{Test description}                        \\
        \hline
        andOr.fo           & Tests boolean operators and their precedence.    \\
        \hline
        and\_sc.fo         & Test if and is properly short circuited.         \\
        \hline
        boolCompare.fo     & Test boolean literals.                           \\
        \hline
        boolLit.fo         & Another test testing if boolean literals.        \\
        \hline
        intNegate.fo       & Test negation of integer values, only test the   \\
                           & simple case ~number not ~(number).  Operator     \\
                           & precedence is also tested.                       \\
        \hline
        muldivide.fo       & Tests the multiplication and division operators. \\
        \hline
        negate2.fo         & Again test simple integer negation.              \\
        \hline
        negate.fo          & Test the case where integer negation is followed \\
                           & by something other than a number.                \\
        \hline
        or\_sc.fo          & Test if or is short circuited.                   \\
        \hline
        tobeornottobe.fo   & Test precedence of boolean operators.            \\
        \hline
        and\_err.fo        & Test a typeerror for and.                        \\
        \hline
        div\_err.fo        & Test a typeerror for div.                        \\
        \hline
        mul\_err.fo        & Test a typeerror for mul.                        \\
        \hline
        neg\_err           & Test a typeerror for negation.                   \\
        \hline
        not\_err.fo        & Test a typeerror for not.                        \\
        \hline
        or\_err.fo         & Test a typeerror for or.                         \\
        \hline
    \end{tabular}

    All the test files are included in appendix \ref{task_1_tests}.


\end{document}
